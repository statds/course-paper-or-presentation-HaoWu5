

Alzheimer's disease (AD) is a serious neurodegenerative condition that affects millions of individuals across the world and the most common cause of dementia. As the average age of individuals in the world increases, the prevalence of AD will continue to grow. The Alzheimer’s Disease Neuroimaging Initiative (ADNI) unites researchers with study data as they work to define the progression of Alzheimer’s disease. ADNI researchers collect, validate, and utilize data, including MRI and PET images, genetics, cognitive tests, Cerebrospinal Fluid (CSF) and blood biomarkers as predictors of the disease. The genetic data from the Alzheimer's Disease Neuroimaging Initiative (ADNI) have been crucial in advancing the understanding of Alzheimer's disease (AD). 


The core element of ADNI genetic data is single-nucleotide polymorphism (SNP), which is always coded by 3 values: 0, 1 and 2.  Considering the fact that the value of SNP does not change over time, we apply a screening procedure \citep{chu2016feature} for varying coefficient models with ultrahigh dimensional longitudinal predictor variables, to find the time-dependent effects of SNPs in response variable Mini Mental State Examination (MMSE), which is a tool that can be used to systematically and thoroughly assess mental status.


