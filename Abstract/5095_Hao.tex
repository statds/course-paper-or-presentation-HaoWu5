%% ====================================================================%%
%%  Developed by Marcos O. Prates, March 2011.
%%
%%	Sample Driver file for use with uconnthesis.sty
%% 	To print this document in the Math Dept,
%%	do the following:
%%  	latex main
%%		bibtex main
%%		latex main (you may have to do this twice at this point)
%%		dvipr main
%%	To run this sample, you will also need the following files
%%		abs.tex, ack.tex,
%%		chap1.tex, chap2.tex, appendix.tex	
%%		thesis.bib	
%% ====================================================================%%
%%
\documentclass[11pt]{report}

%% Include nescessary packages for compilation (THE uconnthesis.sty must be always last)
%% Feel free to add all nescessary package for latex files
%% NOTE: the uconnthesisthesis style file should go AFTER all the other packages
%% You may have conflicts if you call the amsthm.sty package after uconnthesis.sty
\usepackage{amsbsy, amsmath, amssymb, setspace}
\usepackage{natbib}
\usepackage{amscd} % commutative diagrams made easy -  has to be loaded after amsmath
\usepackage{amsthm}  % theorem, lemma, definition, ... -  has to be loaded after amsmath
\usepackage{graphicx} % Imports figures : pdf, jpg, png, tif (one f)
                      % Use the package graphics if you use eps files.
                      % Warning : It is very difficult to use both graphics and graphicx.
\usepackage{rotating}
\usepackage{uconnthesis} % always put it last



\begin{document}
%% Comment out items by inserting a percent % character
%%Set Thesis main title
\title{Time-dependent Effects of SNP on the Alzheimer's Disease Longitudinal Measurements}

%%set author
\name{Hao Wu}



%% You may want to change the next definitions if you are getting
%% a different degree or writing a thesis instead of a dissertation

%% Starts page numbering as i, ii, etc.
%% Generate the first page and include abstract (from abs.tex)
\firstpage
\input abs



%%Generate the list of figures and table (comment if not nescessary)
\figurespagetrue
\tablespagetrue

\singlespacing
\bibliographystyle{plainnat} %natbib has its own special version of plan
\bibliography{ref}



\end{document}
